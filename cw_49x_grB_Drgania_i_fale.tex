\documentclass[a4paper,14pt]{extarticle}  %typ dokumentu

\usepackage[utf8]{inputenc} %rodzaj czcionki
\usepackage{geometry} %poprawienie marginesów
\usepackage{polski} %polskie znaki
\usepackage{multirow} %tabela
\usepackage{graphicx} %tabela
\usepackage{float} %tabela
\usepackage{diagbox} % 2 dane w jednym prostokącie
\usepackage{amsmath, amsthm, amssymb, amsfonts, bbm} % Matma
\usepackage{tikz} %rysowanie
\usepackage{fancyhdr} %headery i footery
\usepackage{circuitikz} %schematy elektryczne
\usepackage{ragged2e} %justowanie tekstu
\usepackage{ulem} %przekreślanie tekstu \xout{text} - kreślenie z lewej na prawo, wiele 
\usepackage{cancel} %przekreślenie tekstu 2 opcja \cancel{text} - pojedyncze skreślenie z prawej na lewą
%\usepackage{subcaption}
%\usepackage{caption}
%\usepackage{enumitem}
%\usepackage{blindtext} 


%DO SPRAWDZENIA (używane przez profesjonalistów)
% \usepackage{indentfirst}
% \usepackage{textcomp}
% \usepackage{exscale}
% \usepackage{hyperref}
% \usepackage{eurosym}
% \usepackage{mathrsfs}
% \usepackage{amscd}

\graphicspath{{pictures/}}
\geometry{margin=0.7in}
\pagestyle{fancy}
\fancyhf{}
\cfoot{Strona \thepage}
\rhead{Strona \thepage}

%-----------------------PRZYDATNE LINKI----------------------------------
%link do tworzenia tabeli https://tablesgenerator.com
%symbole matematyczne: https://oeis.org/wiki/List_of_LaTeX_mathematical_symbols
%narzedzia matematyczne: https://en.wikibooks.org/wiki/LaTeX/Mathematics
%krótkie podpowiedzi http://www.mif.pg.gda.pl/homepages/sylas/students/wdi/doc/latex-sciaga.html
%symbole do schematów: http://texdoc.net/texmf-dist/doc/latex/circuitikz/circuitikzmanual.pdf
%----------------------------------------------------------------------


%-----------------------SEKCJA DANYCH----------------------------------
\def\tytul{Drgania i Fale} %<<< tytuł ćwiczenia
\def\nrcw{6} %<<< numer cwiczenia
\def\data{\today} %<< data wykonania
\def\prowadzacy{dr inż. Ewa Frączek} %<<<prowadzący
\def\tworcy{Byczko Maciej\\Maziec Michał\\Pomarański Maciej} %<<< autorzy
\def\nrgrupy{C} %<<<numer grupy
\def\zajinfo{Tydzień parzysty\\ Godzina 11:15-13:00} %<<<informacje dotyczące zajęć

\lhead{Ćwiczenia z fizyki - \tytul} %typ dokumentu tj Sprawozdanie, zadania itp.

%JEZELI COS JESZCZE POTRZEBA W TEJ SEKCJI TO POINFORMOWAĆ!!!
%----------------------------------------------------------------------

%-----------------------SEKCJA FORMATOWANIA----------------------------
% \textbf{pogrubienie}  \textit{kursywa}    \underline{podkreślenie}

%----------------------------------------------------------------------



\begin{document}

%-------------------------------------TABELA-DANYCH--------------------------------------------------
    \begin{table}[H]
        \centering
        \resizebox{\textwidth}{!}{
        \begin{tabular}{|c|c|c|}\hline
            \begin{tabular}[c]{@{}c@{}}                     \tworcy     \end{tabular}&
            \begin{tabular}[c]{@{}c@{}}Prowadzący:\\        \prowadzacy \end{tabular}&
            \begin{tabular}[c]{@{}c@{}}Numer ćwiczenia\\    \nrcw       \end{tabular}        \\\hline
            \begin{tabular}[c]{@{}c@{}}Grupa nr.\\          \nrgrupy    \end{tabular}& 
            \begin{tabular}[c]{@{}c@{}}Temat ćwiczenia:\\   \tytul      \end{tabular}& Ocena:\\\hline
            \begin{tabular}[c]{@{}c@{}}                     \zajinfo    \end{tabular}&
            \begin{tabular}[c]{@{}c@{}}Data wykonania:\\    \data       \end{tabular}&       \\\hline
        \end{tabular}}
    \end{table}
%----------------------------------------------------------------------------------------------------

    \centering
    \section{Zadania do zrobienia}
        \subsection{Zadanie 1}
            \subsubsection{Polecenie}
                Zad.3(P) Ile wynosi stosunek energii kinetycznej do potencjalnej ciała wykonującego drgania harmoniczne 
                kosinusoidalne dla chwili czasu t=$\frac{T}{6}$, jeżeli faza początkowa wynosi zero?\\ 
                Ile będzie wynosił ten sam stosunek energii dla drgania harmonicznego sinusoidalnego?
            \subsubsection{Rozwiązanie}
                \textbf{Wzory:}\\
                $E_k(t)=\frac{1}{2}mv^2(t)\Leftrightarrow E_k(t)=\frac{1}{2}m[x_0\omega _0\sin(\omega_0t+\phi)]^2$\\
                $E_p(t)=\frac{1}{2}kx^2(t)\Leftrightarrow E_p(t)=\frac{1}{2}k[x_0\cos(\omega_0t+\phi)]^2$\\
                \textbf{Obliczenia:}\\
                $\frac{E_k(t)}{E_p(t)}=\frac{\frac{1}{2}m[x_0\omega _0\sin(\omega_0t)]^2}{\frac{1}{2}k[x_0\cos(\omega_0t)]^2}$\\
                $\frac{E_k(t)}{E_p(t)}=\frac{mx_0^2\omega_0^2[\sin(\frac{2\pi}{T}*\frac{T}{6}]^2}{kx_0^2[\cos(\frac{2\pi}{T}*\frac{T}{6})]^2}$\\
                $\frac{E_k(t)}{E_p(t)}=\frac{m\omega _0^2[\sin(\frac{\pi}{3})]^2}{k[\cos(\frac{\pi}{3})]^2}$\\
                $\frac{E_k(t)}{E_p(t)}=\frac{k}{k}*[\tan(\frac{\pi}{3})]^2 = 3$\\
                Dla drgania harmonicznego sinusoidalnego stosunek będzie wynosił $\frac{\sqrt{3}}{3}$ ponieważ:\\
                $\frac{E_k(t)}{E_p(t)}=\frac{k}{k}*[\cot(\frac{\pi}{3})]^2 = \frac{1}{3}$\\
                \clearpage
        \subsection{Zadanie 2}
            \subsubsection{Polecenie}
                Zad.9. Wartości amplitud wymuszonych drgań harmonicznych są równe dla dwóch częstości siły wymuszającej: 
                $\omega_1$=400 $\frac{rad}{s}$ oraz $\omega_2$=600 $\frac{rad}{s}$. 
                Wyznacz częstość $\omega_{rez}$, dla której amplituda drgań wymuszonych osiągnie maksymalną wartość.
            \subsection{Rozwiązanie}
                \textbf{Wzory:}\\
                Mechaniczne drgania wymuszone:$F(t)=F_0\cos(\omega t)$\\
                $\frac{d^2x(t)}{dt^2}+2\beta\frac{dx(t)}{dt}+\omega_0^2x=f_0\cos(\omega t)$\\
                Współczynnik tłumienia:$2\beta=\frac{b}{m}$\\
                Częstość oscylatora nietłumionego:$omega_0^2=\frac{k}{m}$\\
                $f_0=\frac{F_0}{m}$\\
                Częstość rezonansowa:$\omega_r=\sqrt{\omega_0^2-2\beta^2}$\\
                Amplituda wymuszona:$A(\omega)=\frac{F_0}{m\sqrt{(\omega_0^2-\omega^2)^2+4\beta^2\omega^2}}$
                Amplituda rezonansowa:$A_r=\frac{f_0}{2\beta(\omega_0^2-\beta^2)}$\\
                \textbf{Obliczenia:}\\
                $A(\omega)=\frac{F_0}{m\sqrt{(\omega_0^2-\omega^2)^2+4\beta^2\omega^2}}$\\
                $\omega_r=\sqrt{\omega_0^2-2\beta^2}$    $A(\omega_1)=A(\omega_2)$\\
                $(\omega_0^2-\omega_1^2)^2+4\beta^2\omega_1^2=(\omega_0^2-\omega_2^2)^2+4\beta^2\omega_2^2$\\
                $\cancel{\omega^4_0}-2\omega^2_0\omega^2_1+\omega^4_1+4\beta2\omega^2_1=\cancel{\omega^4_0}-2\omega^2_0\omega^2_2+\omega^4_2+4\beta2\omega^2_2$\\
                $\omega^4_1-2\omega^2_1(\omega^2_0-2\beta^2)=\omega^4_2-2\omega^2_2(\omega^2_0-2\beta^2)$\\
                $\omega^4_1-2\omega^2_1\omega^2_r=\omega^4_2-2\omega^2_2\omega^2_r$\\
                Wyznaczenie $\omega_r$:\\
                $-2\omega^2_1\omega^2_r+2\omega^2_2\omega^2_r=\omega^4_2-\omega^4_1$\\
                $\omega^2_r(2\omega^2_2-2\omega^2_1)=(\omega^2_2-\omega^2_1)(\omega^2_2+\omega^2_1)$\\
                $\omega_r=\sqrt{\frac{(\omega^2_2-\omega^2_1)(\omega^2_2+\omega^2_1)}{2(\omega^2_2-\omega^2_1)}}$\\
                $\omega_r=\sqrt{\frac{\omega^2_1+\omega^2_2}{2}}$\\
                $\omega_r=\sqrt{\frac{600^2+400^2}{2}}=510\frac{rad}{s}$\\

    \section{Wymyślone zadanie z fali kulistej}
        \subsection{Polecenie}
        \subsection{Rozwiązanie}
        \textbf{Wzory:}\\
        Natężenie fali: $I=\frac{P_{zr}}{4\Pi r^2}$\\
        Prawo odwrotnych kwadratów: $\frac{I_1}{I_2}=\frac{r_2^2}{r_1^2}$\\
        $I=\frac{1}{2}\rho v \omega ^2 s_m^2$\\
        $\Phi (r,t)=\frac{s_0}{r}\sin (kr-\omega t)$\\
        Interferencja kostruktywna:$L_2-L_1=m\lambda$\\
        Interferencja destruktywna:$L_2-L_1=(m+\frac{1}{2})\lambda$\\
        $m \in \mathbb{N} $



\end{document}