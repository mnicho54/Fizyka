\documentclass[14pt]{extarticle}
\usepackage[utf8]{inputenc}
\usepackage{polski}
\usepackage{amsmath}
\usepackage{geometry}
\usepackage{graphicx}
\usepackage{multirow}
\usepackage{float}
\usepackage{esvect}
\graphicspath{{pictures/}}
\geometry{margin=0.7in}
\def\tytul{Zasada zachowania energii}

\begin{document}
\begin{table}[H]
    \centering
    \resizebox{\textwidth}{!}{
    \begin{tabular}{|c|c|c|}
    \hline
    
    \begin{tabular}[c]{@{}c@{}}Byczko Maciej\\Maziec Michał\\Pomarański Maciej\end{tabular}&
    \begin{tabular}[c]{@{}c@{}}Prowadzący:\\ dr inż. Ewa Frączek\end{tabular} &
    \begin{tabular}[c]{@{}c@{}}Numer ćwiczeń\\
    %----------------------------------------
         %<<<tutaj wpisz numer ćwiczenia
    %----------------------------------------
    \end{tabular} \\ \hline
    \begin{tabular}[c]{@{}c@{}}Grupa\\
    %----------------------------------------
        C %<<<tutaj wpisz numer grupy
    %----------------------------------------    
    \end{tabular} & \begin{tabular}[c]{@{}c@{}}Temat ćwiczenia:\\\tytul
    \end{tabular} &3\\ \hline\begin{tabular}[c]{@{}c@{}}Tydzień parzysty\\ Godzina 11:15-13:00\end{tabular}&\begin{tabular}[c]{@{}c@{}}Data wykonania ćwiczenia:\\
    %----------------------------------------
    17 marca 2020 %<<<tutaj wpisz datę
    %----------------------------------------
    \end{tabular} &\begin{tabular}[c]{@{}c@{}}Kod grupy:\\E07-50d\end{tabular}\\ \hline\end{tabular}}\end{table}
    \centering
    \section{Zadanie}
    \begin{flushleft}
        Naukowcy potrzebują twojej pomocy! Chcą zbudować ze 144 sprężyn trampolinę, która ma wytrzymać uderzenie samochodu ważącego 1500kg zrzuconego z 45 metrów.
        Sprężyny mogą maksymalnie rozciągnąć się na 4,5m. w przeciwnym wypadku samochód uderzy w ziemię.
        Policz współczynnik sprężystości \textbf{\underline{jednej}} sprężyny jaką potrzebują naukowcy.
        Przyjmij współczynnik grawitacji 10$\frac{m}{s^2}$
    \end{flushleft}
    \begin{figure}[H]
        \centering
        \includegraphics{trampolina}
        \caption{Rysunek pomocniczy}
    \end{figure}
    \clearpage
    \section{Zadanie}
    \begin{flushleft}
        Dwóch kolegów trenują z piłkami nożnymi, w pewnym momencie obydwaj kopią swoje piłki tak że pierwsza leci pod kątem $\alpha _1$=45$^\circ$ a druga pod kątem $\beta _1$=0$^\circ$,
        następnie zderzają się i lecą dalej gdzie pierwsza leci pod kątem $\beta _2$=30$^\circ$.
        Pierwszy z nich kopnął swoją piłkę z prędkością $v_1$=50$\frac{km}{h}$ a drugi z nich z prędkością $v_2$=40$\frac{km}{h}$.
        Znajdź prędkości końcowe tych piłek. Masy są identyczne.
    \end{flushleft}
    \begin{figure}[H]
        \centering
        \includegraphics{pilki}
        \caption{Rysunek pomocniczy}
    \end{figure}
    \clearpage
    \section{Rozwiązania}
    \subsection{Zadanie 1}
    \Large
        \(mgh=\frac{kx^2}{2} \rightarrow k=\frac{2mgh}{x^2}\)\\
        \(k=\frac{2\ast 1500 \ast 10\ast 45}{(4.5)^2} \rightarrow k=66666.(6)\frac{N}{m}\)\\
        k\footnotesize sprężyny \Large\(=\frac{66666.(6)}{144} \rightarrow\)k\footnotesize sprężyny \LARGE=\underline{\(462.(962)\frac{N}{m}\)}
    \subsection{Zadanie 2}
    \large
    $m_1\vv{v_1}+m_2\vv{v_2}=m_1\vv{u_1}+m_2\vv{u_2} \rightarrow \vv{v_1} + \vv{v_2} = \vv{u_1} + \vv{u_2}$\\
    $\frac{m_1v_1^2}{2}+\frac{m_2v_2^2}{2}=\frac{m_1u_1^2}{2}+\frac{m_2u_2^2}{2} \rightarrow v_1^2 + v_2^2 = u_1^2 + u_2^2 $
    
    \begin{equation*}
        \begin{cases}
           v_1\cos\alpha _1 + v_2\cos\beta _1 = u_1\cos\alpha _2 + u_2\cos\beta _2 &\\
            -v_1\sin\alpha _1 + v_2\sin\beta _1 = u_1\sin\alpha _2 - u_2\sin\beta _2 &\\
           v_1^2 + v_2^2 = u_1^2 + u_2^2 \ \ [z \ danych \ mamy:v_1=1.25v_2 \ bo: 50/40=1.25] &
        \end{cases}
    \end{equation*}
    \begin{equation*}
        \begin{cases}
            \cos\alpha _2 = \frac{v_1(\frac{\sqrt{2}}{2}+1)-\frac{\sqrt{3}}{2}u_2}{u_1}&(1)\\
            \sin\alpha _2 = \frac{\frac{\sqrt{2}}{2}v_1+\frac{\sqrt{3}}{2}u_2}{u_1}&(2)\\
            u_1^2 = v_1^2 + v_2^2 - u_2^2&(3)
        \end{cases}
    \end{equation*}
    $\sin^2\alpha _2 + \cos^2\alpha _2 = 1\ \vv{z \ (1)i(2):}
    \left(\frac{v_1(\frac{\sqrt{2}}{2}+1)-\frac{\sqrt{3}}{2}u_2}{u_1}\right)^2 + \left(\frac{\frac{\sqrt{2}}{2}v_1+\frac{\sqrt{3}}{2}u_2}{u_1}\right)^2 = 1 \ (4)$
    \begin{equation*}
        \begin{cases}
            u_1^2 = (2+\sqrt{2})v_1^2 + \sqrt{3}v_1u_2 + \frac{6}{4} u_2^2&(4)\\
            u_1^2 = v_1^2 + v_2^2 - u_2^2&(3)
        \end{cases}
    \end{equation*}
        Przyrównanie: $2.25v_1^2 - u_2^2=(2+\sqrt{2})v_1^2 + \sqrt{3}v_1u_2 + \frac{6}{4} u_2^2$\\
        $ 0=(\sqrt{2}-0.25)v_1^2+\sqrt{3}v_1u_2+\frac{10}{4}u_2^2$\\

        
            
\end{document}