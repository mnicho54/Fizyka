\documentclass[14pt]{extarticle}
\usepackage[utf8]{inputenc}
\usepackage{polski}
\usepackage{amsmath}
\usepackage{geometry}
\usepackage{graphicx}
\usepackage{multirow}
\usepackage{float}

\geometry{margin=0.7in}
\def\tytul{Zasada zachowania energii}

\begin{document}
\begin{table}[H]
    \centering
    \resizebox{\textwidth}{!}{
    \begin{tabular}{|c|c|c|}
    \hline
    
    \begin{tabular}[c]{@{}c@{}}Byczko Maciej\\Maziec Michał\\Pomarański Maciej\end{tabular}&
    \begin{tabular}[c]{@{}c@{}}Prowadzący:\\ dr inż. Ewa Frączek\end{tabular} &
    \begin{tabular}[c]{@{}c@{}}Numer ćwiczeń\\
    %----------------------------------------
         %<<<tutaj wpisz numer ćwiczenia
    %----------------------------------------
    \end{tabular} \\ \hline
    \begin{tabular}[c]{@{}c@{}}Grupa\\
    %----------------------------------------
        C %<<<tutaj wpisz numer grupy
    %----------------------------------------    
    \end{tabular} & \begin{tabular}[c]{@{}c@{}}Temat ćwiczenia:\\\tytul
    \end{tabular} &3\\ \hline\begin{tabular}[c]{@{}c@{}}Tydzień parzysty\\ Godzina 11:15-13:00\end{tabular}&\begin{tabular}[c]{@{}c@{}}Data wykonania ćwiczenia:\\
    %----------------------------------------
    17 marca 2020 %<<<tutaj wpisz datę
    %----------------------------------------
    \end{tabular} &\begin{tabular}[c]{@{}c@{}}Kod grupy:\\E07-50d\end{tabular}\\ \hline\end{tabular}}\end{table}
    \centering
    \section{Zadanie}
    \begin{flushleft}
        Naukowcy potrzebują twojej pomocy! Chcą zbudować ze 144 sprężyn trampolinę, która ma wytrzymać uderzenie samochodu ważącego 1500kg zrzuconego z 45 metrów.
        Sprężyny mogą maksymalnie rozciągnąć się na 4,5m. w przeciwnym wypadku samochód uderzy w ziemię.
        Policz współczynnik sprężystości \textbf{\underline{jednej}} sprężyny jaką potrzebują naukowcy.
        Przyjmij współczynnik grawitacji 10$\frac{m}{s^2}$
    \end{flushleft}
    \section{Zadanie}
    \begin{flushleft}
        tutaj będzie zadanie 2
    \end{flushleft}
    %\clearpage
    \section{Rozwiązania}
    \subsection{Zadanie 1}
    \LARGE
        \(mgh=\frac{kx^2}{2} \rightarrow k=\frac{2mgh}{x^2}\)\\
        \(k=\frac{2\ast 1500 \ast 10\ast 45}{(4.5)^2} \rightarrow k=66666.(6)\frac{N}{m}\)\\
        k\footnotesize sprężyny \LARGE\(=\frac{66666.(6)}{144} \rightarrow\)k\footnotesize sprężyny \LARGE=\underline{\(462.(962)\frac{N}{m}\)}
    \subsection{Zadanie 2}
    
\end{document}